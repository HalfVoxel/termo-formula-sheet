% !TEX encoding = UTF-8 Unicode
\documentclass[10pt,landscape]{article}
\usepackage[T1]{fontenc}
\usepackage[utf8]{inputenc} % set input encoding (not needed with XeLaTeX)

\usepackage{multicol}
\usepackage{calc}
\usepackage{ifthen}
\usepackage[landscape]{geometry}
%\usepackage[landscape]{geometry}
\usepackage{hyperref}
\usepackage{siunitx}
\usepackage{microtype}
\usepackage{todonotes}

\usepackage[fleqn]{mathtools}

\setlength{\mathindent}{0pt}

\usepackage{titlesec}
  
% To make this come out properly in landscape mode, do one of the following
% 1.
%  pdflatex latexsheet.tex
%
% 2.
%  latex latexsheet.tex
%  dvips -P pdf  -t landscape latexsheet.dvi
%  ps2pdf latexsheet.ps


% If you're reading this, be prepared for confusion.  Making this was
% a learning experience for me, and it shows.  Much of the placement
% was hacked in; if you make it better, let me know...


% 2008-04
% Changed page margin code to use the geometry package. Also added code for
% conditional page margins, depending on paper size. Thanks to Uwe Ziegenhagen
% for the suggestions.

% 2006-08
% Made changes based on suggestions from Gene Cooperman. <gene at ccs.neu.edu>


% To Do:
% \listoffigures \listoftables
% \setcounter{secnumdepth}{0}


% This sets page margins to .5 inch if using letter paper, and to 1cm
% if using A4 paper. (This probably isn't strictly necessary.)
% If using another size paper, use default 1cm margins.
\geometry{top=0.8cm,left=0.8cm,right=0.8cm,bottom=0.8cm}
\geometry{a4paper}

\usepackage[absolute]{textpos}
\usepackage{vruler}

% Turn off header and footer
\pagestyle{empty}
 
\newcommand{\unit}[1]{
\;[\SI{}{#1}]
}


%\titleformat{\section}
 % {\normalfont\Large\bfseries}{\thesection}{1em}{}[{}]
  
% Redefine section commands to use less space
\makeatletter
\renewcommand{\section}{\@startsection{section}{1}{0mm}%
                                {-1ex plus -.5ex minus -.2ex}%
                                {0.5ex plus .2ex}%x
                                {\normalfont\large\bfseries}
                                }
\renewcommand{\subsection}{\@startsection{subsection}{2}{0mm}%
                                {-1explus -.5ex minus -.2ex}%
                                {0.5ex plus .2ex}%
                                {\normalfont\normalsize\bfseries}}
\renewcommand{\subsubsection}{\@startsection{subsubsection}{3}{0mm}%
                                {-1ex plus -.5ex minus -.2ex}%
                                {1ex plus .2ex}%
                                {\normalfont\small\bfseries}}
\makeatother


% Don't print section numbers
\setcounter{secnumdepth}{0}


\setlength{\parindent}{0pt}
\setlength{\parskip}{0pt plus 0.5ex}




  
% -----------------------------------------------------------------------

\def\dbar{{\mathchar'26\mkern-12mu d}}

\begin{document}

\begin{textblock*}{200mm}(5mm,-1mm)
$$
\vbox{
\def\1{\vrule height 0pt depth 2pt}
\def\2{\vrule height 0pt depth 4pt}
\def\3{\vrule height 0pt depth 6pt}
\def\4{\vrule height 0pt depth 8pt}
\def\ruler#1#2#3{\leftline{$\vcenter{\hrule\hbox{\4#1}}\,\,\rm#2\,{#3}$}}
%\def\\#1{\hbox to .125in{\hfil#1}}
%\def\8{\\\1\\\2\\\1\\\3\\\1\\\2\\\1\\\4}

%\def\\#1{\hbox to 10pt{\hfil#1}}
%\def\8{\\\1\\\1\\\1\\\1\\\2\\\1\\\1\\\1\\\1\\\4}

%\def\\#1{\hbox to 10dd{\hfil#1}}
%\def\8{\\\1\\\1\\\1\\\1\\\2\\\1\\\1\\\1\\\1\\\4}

\def\\#1{\hbox to 1mm{\hfil#1}}
\def\8{\\\1\\\1\\\1\\\1\\\2\\\1\\\1\\\1\\\1\\\3}
\def\9{\\\1\\\1\\\1\\\1\\\2\\\1\\\1\\\1\\\1\\\4}

\ruler{\8\8\8\8\8\8\8\8\8\9\8\8\8\8\8\8\8\8\8\8}{20}{cm}
}$$
\end{textblock*}

\raggedright
\footnotesize
\scriptsize
\begin{multicols}{5}


% multicol parameters
% These lengths are set only within the two main columns
%\setlength{\columnseprule}{0.25pt}
\setlength{\premulticols}{1pt}
\setlength{\postmulticols}{1pt}
\setlength{\multicolsep}{1pt}
\setlength{\columnsep}{2pt}

\begin{center}
     \Large{Termodynamik - Slafs}\\
     \tiny{Aron Granberg, Daniel Kempe, Mårten Wiman}
\end{center}

$$$$
%\begin{multicols}{2}


%\section{Ideala och icke-ideala gaser}


\section{Utvidgning}
$\kappa = -\frac{1}{V}\left(\frac{\partial V}{\partial p}\right)_T \unit{\per\pascal}$

Isobar volymutvidgningskoefficient\\
$\alpha_V = \frac{1}{V}\left( \frac{\partial V}{\partial T} \right)_p \unit{\per\kelvin}$


Relativa volymändringen\\
$\frac{dV}{V} = -\kappa \cdot dp + \alpha_V \cdot dT$ 

% Lägg till utvidgninskoeff.




\section{Kinetisk gasteori}
$m = \text{massan per partikel} \unit{\kilogram}$

Molara massan\\
$M = mN_A$ 


$\nu R = Nk_B$

$ n = \frac{N}{V}$

$v_p = \sqrt{2} \cdot \sqrt{\frac{k_B T}{m}}$

$\langle v \rangle = \sqrt{\frac{8}{\pi}} \cdot \sqrt{\frac{k_BT}{m}}$

$v_{rms} = \sqrt{\langle v^2 \rangle} = \sqrt{3} \cdot \sqrt{\frac{k_BT}{m}}$

$\langle E_k \rangle = \frac{3k_BT}{2}$

Ekvipartitionsprincipen\\
$U = Nk_BT \cdot \frac{1}{2} \cdot (\# \text{frihetsgrader}) \unit{\joule}$ 


Energi i enatomig gas\\
$U = N \frac{m \langle v^2 \rangle}{2} = \frac{3}{2} Nk_BT \unit{\joule}$ 


Notera $Nk_BT = pV$

$pV = \frac{2}{3}U$

$$$$

Medelfri väg\\
$l = \frac{k_BT}{\sqrt{2}\pi d^2 p} = \frac{1}{n \sigma \sqrt{2}}$  


Där $d = \text{partikelns diameter}$

%\begin{align*}
%Medelfri väg}\\
%l = <v>\tau = \frac{1}{n \sigma \sqrt{2}}  & \text{

%\end{align*}

Stöttal\\
$\nu^* = \frac{p}{\sqrt{2 \pi mk_BT}} = \frac{1}{4}n \langle v \rangle \unit{\per\second \per\square\meter}$ 


Maxwell-Boltzmanns hastighetsfördelning\\
$\displaystyle n(v) = \text{K} \cdot v^2 \cdot e^{-\frac{mv^2}{2k_BT}}$ 

om $\int n(v) = \frac{N}{V}$, dvs om normaliserat\\
$K = 4\pi n \left(\frac{m}{2\pi k_B T} \right)^\frac{3}{2}$ 


\section{Värme}
Energi för att förändra temp.\\
$\Delta Q = m c \Delta T \unit{\joule}$ 


Molar isokor värmekapacitet ideal gas\\
$C_V = \frac{1}{\nu} \frac{dU}{dT} \unit{\joule\per\mol\per\kelvin}$ 

Enatomig ideal gas har\\
$C_V = \frac{3}{2}R$

Molar isobar värmekapacitet ideal gas\\
$C_p = C_V + R \unit{\joule\per\mol\per\kelvin}$ 

Molar värmekapacitet fast kropp\\
$C_m = 3R \unit{\joule\per\mol\per\kelvin}$ 




%\section{Tryck}
%Partialtryck\\
%$p = \sum p_i$ 


\section{Adiabatiska processer}

%$\Delta S = 0; \Delta Q = 0$

$C_p = \text{isobara molara värmekapaciteten}$

$C_V = \text{isokora molara värmekapaciteten}$

% TODO Lägg till C_p = C_V + R för ideala gaser

$\gamma = \frac{C_p}{C_V} = \frac{c_p}{c_V}$

$pV^\gamma = \text{konst.}$

$Tp^{(1-\gamma)/\gamma} = \text{konst.}$

$TV^{\gamma-1} = \text{konst.}$                                                                                                                                                                                                                                    

Adiabatiskt arbete på en gas\\
$\displaystyle W = -\int_0^1 pdV = \frac{p_1V_1 - p_2V_2}{1-\gamma}$

\section{Matematik}
Sfär: $A = 4\pi r^2; V = \frac{4\pi r^3}{3}$

\section{Värmetransport}
$\lambda = \text{Värmekonduktivitet}$

$\alpha = \text{Värmeövergångskoefficient}$

Ledning\\
$U = \frac{\lambda}{d} \unit{\watt\per\kelvin\per\square\meter}$ 


Konvektion\\
$U = \alpha \unit{\watt\per\kelvin\per\square\meter}$ 


Värmemotstånd\\
$\frac{1}{U} = \sum \frac{1}{U_i}$ 


Värmeflöde\\
$\Phi = UA \left(T_i - T_u \right)$ 

 
Kom ihåg: Vid jämvikt är värmeflödet konstant, och i t.ex en vägg är värmeflödet konstant genom hela väggen.

\section{Första huvudsatsen}
Arbete på en gas\\
$\dbar W = -pdV$ 


Energiutbyte med omgivningen\\
$\dbar Q = dU + pdV$ 


Derivatan av inre energi\\
$dU = \dbar Q + \dbar W = \dbar Q - pdV$ 


Vid isokor process\\
$dU = \nu C_V dT$ 


%Entalpi\\
%$H = U + pV$ 


Arbete på en gas\\
$\displaystyle W = -\int_{1}^{2} pdV$ 


Isotermt kompressionsarbete på en gas\\
$W_T = -\nu RT \ln \left(\frac{V2}{V_1}\right)$ 


Isobart kompressionsarbete på en gas\\
$W_p = -p_2 (V_2 - V_1)$ 


Isokort arbete på en gas\\
$W_V = 0$


% molar värmekapacitet

\section{Andra huvudsatsen}
Tillförs $dQ$ reversibelt till ett system så är
$dS = \frac{dQ}{T}$

Reversibel process i slutet system $\Delta S = 0$\\
Irreversibel process i slutet system $\Delta S > 0$

$\Delta S = \nu C_V \cdot \ln \frac{T_2}{T_1} + \nu R \cdot \ln \frac{V_2}{V_1}$

\section{Övrigt om entropi}
$T=0 \Rightarrow S = 0$

$W =$ antal möjliga mikroskopiska tillstånd\\
$S = k_B\ln W$

Om $S_A$ är entropi för system A och $S_B$ entropi för system B så har
$S_A$ och $S_B$ sett som ett enda system entropin\\
$S_{A \cup B} = S_A + S_B$

$$$$

Entalpi\\
$H = U + pV$\\
$dH = dU + pdV + Vdp$

Fria energin (Helmholtz funktion)\\
$F = U -TS$\\
$dF = dU - TdS - SdT$

Fria entalpin (Gibbs funktion)\\
$G = F + pV$\\
\textcolor{orange}{ska vi kunna detta?}
$dG = -SdT + Vdp + \mu N$

$$$$
Vid isoterm process så är\\
$\dbar W = dF$

Vid fasövergång är $H$ ej kontinuerlig (med avseende på temperatur), $G$ är kontinuerlig men dess derivata är inte det

$H = G + TS$

\section{Carnotprocesser}
%Index H avser varma (hot) delen i processen, C avser kalla (cold) delen
$T_H \ge T_C$

Var noga med tecken

$Q_H$ Värme som reservoaren vid $T_H$ avger\\
$Q_C$ Värme som reservoaren vid $T_C$ avger\\
$W$ Arbete som tillförs processen

$\frac{Q_H}{T_H} = -\frac{Q_C}{T_C}$ \hfill Notera tecken

$-W = Q_H + Q_C$ {\tiny(termer kan vara negativa)}

$|W| = |Q_H| - |Q_C|$

%$\frac{Q_H}{T_H} = \frac{Q_C}{T_C}$

$\eta = \frac{Q_H - Q_C}{Q_H} = \frac{T_H - T_C}{T_H}$

\end{multicols}

%\vskip\medskipamount % or other desired dimension
%\leaders\vrule width \textwidth\vskip0.4pt % or other desired thickness
%\vskip\medskipamount % ditto
%\nointerlineskip

\begin{textblock*}{299mm}(0mm,85mm)
%\noindent\makebox[\linewidth]{\rule{\textwidth}{1pt}} 
\end{textblock*}

\begin{multicols}{3}
\section{Konstanter}
\begin{tabular}{l l r l}
Massenhet & u & $1.66054 \cdot 10^{-27}$ & $\SI{}{\kilogram}$\\
Avogadros & $N_A$ & $6.02214 \cdot 10^{23}$ & $\SI{}{\per\mol}$\\
Boltzmanns & $k_B$ & $1.38065 \cdot 10^{-23}$ & $\SI{}{\joule\per\kelvin}$\\
Gaskonstanten & R & $8.3145$ & $ \SI{}{\joule\per\mol\per\kelvin}$\\
Stefan-Boltzmanns & $\sigma$ & $5.6704 \cdot 10^{-8}$ & $\SI{}{\watt \per\square\meter \per\raiseto{4}\kelvin}$\\
%0 \SI{}{\degreeCelsius} & 0 \SI{}{\degreeCelsius} & 273.15 & \SI{}{\kelvin}\\
Plancks & $h$ & $6.62607 \cdot 10^{-34}$ & \SI{}{\joule\second}\\
Ljushastigheten & $c$ & 299 792 458 & \SI{}{\meter\per\second}\\
\end{tabular}

\section{Vettiga värden}
\begin{tabular}{l r l}
Arbete vid sömn & 1 & \SI{}{\watt\per\kilogram}\\
Lätt arbete utvecklar vid 25\% eff. & 55-75 & \SI{}{\watt}\\
Energibehov människa (3000 kcal) & 12 & \SI{}{\mega\joule\per\day}\\
Jordens radie & $6.4 \cdot 10^{6}$ & $\SI{}{\meter}$\\
Månens radie & $1.7 \cdot 10^6$ & $\SI{}{\meter}$\\
Sveriges area & $4.5 \cdot 10^{11}$ & \SI{}{\square\meter}\\
%Solarkonstanten & 1366 & \SI{}{\watt\per\meter^2}\\
%Värmekapacitet  $c_{vatten}$ & 4.18 & \SI{}{\kilo\joule\per\kilogram\per\kelvin}\\
Värmekapacitet  $c_{luft}$ & 1.007 & \SI{}{\kilo\joule\per\kilogram\per\kelvin}\\
Energidensitet Li-ion batteri & $0.3-0.9$ & \SI{}{\mega\joule\per\kilogram}\\
Energidensitet trä & $16$ & \SI{}{\mega\joule\per\kilogram}\\
Energidensitet kol & $24$ & \SI{}{\mega\joule\per\kilogram}\\
Energidensitet fett & $37$ & \SI{}{\mega\joule\per\kilogram}\\
Energidensitet bensin & $44$ & \SI{}{\mega\joule\per\kilogram}\\
Energidensitet uran & $8.1 \cdot 10^7$ & \SI{}{\mega\joule\per\kilogram}\\
Sveriges elkonsumption & $1.5 \cdot 10^{10}$ & $\SI{}{\watt}$\\
Världens elkonsumption & $2.1 \cdot 10^{12}$ & $\SI{}{\watt}$\\
Sveriges energikonsumption & $7.4 \cdot 10^{10}$ & $\SI{}{\watt}$\\
Världens energikonsumption & $1.5 \cdot 10^{13}$ & $\SI{}{\watt}$\\
\end{tabular}

\parbox{.34\linewidth}{
\subsection{Kemi}
\begin{tabular}{l r}
Atom & Atomnummer\\
Kol & 6\\
Kväve & 7\\
Syre & 8\\
Neon & 10\\
\end{tabular}

Glöm inte bort att molekyler är flera atomer
}
\parbox{.32\linewidth}{
\begin{tabular}{l r }
Substans & $C_V / R$\\
\hline
$He$ & 1.52\\
$H_2$ & 2.44\\
$N_2$ & 2.49\\
$O_2$ & 2.51\\
$CO$ & 2.53\\
\end{tabular}
}
\parbox{.32\linewidth}{
\begin{tabular}{l r }
Ämne & $\gamma$\\
\hline
Luft & 1.4\\
$H_2$ & 1.4\\
$CO_2$ & 1.3\\
$H_2O$ & 1.3\\
&\\
\end{tabular}
}

%\parbox{.40\linewidth}{
%\begin{tabular}{l r}
%Material & $\lambda$ \unit{\watt\per\meter\per\kelvin}\\
%\hline
%Luft & 0.024\\
%Vatten & 0.60\\
%Järn & 75\\
%Koppar & 390\\
%Furuträ & 0.14\\
%Glas & 0.9\\
%Betong & 1.7\\
%Tegel & 0.6\\
%Mineralull & 0.038\\
%\end{tabular}
%}
%\;\;\;\;\;\;\;\;
\parbox{.40\linewidth}{
\begin{tabular}{l r}
Ämne & Densitet \unit{\kilogram\per\meter\cubed}\\
\hline
Kol & 1050\\
Vatten & 1000\\
Järn & 7844\\
Luft & 1.275\\
Helium & 0.1785\\
Väte & 0.0899\\
Nysnö & 60\\
Packad snö & 400\\
Is & 850\\
\end{tabular}
}


\section{Tillståndsekvationer för gaser}
$M = \text{molara massan} \unit{\kilogram\per\mol}; m = \text{totala massan i systemet} \unit{\kilogram}$

$\displaystyle \rho = \frac{m}{V}; p = \frac{\rho R T}{M} = \frac{Nk_BT}{V} = \frac{\nu R T}{V}; \nu = \frac{m}{M}$

$\displaystyle b \approx \text{molekylens volym}; a \approx \text{växelverkan mellan partiklar}$

$\displaystyle p = \frac{Nk_BT}{V - Nb} - a\left( \frac{N}{V} \right)^2$\hfill Van der Waals tillståndsekvation

$b_0 = bN_A; a_0 = aN_A^2; v = \frac{V}{\nu}$

$\displaystyle \left(p + \frac{a_0}{v^2} \right)\cdot (v - b_0) = RT$\hfill Van der Waals tillståndsekvation

\section{Strålning}
$\varepsilon = \text{emissivitet}; \alpha = \text{absorptionsfaktor}$\\
$\rho = \text{reflexionsfaktor}; \tau = \text{transmissionsfaktor}$

$\nu = \text{frekvens} = \frac{c}{\lambda}$

$\text{Svartkropp} \Rightarrow \varepsilon = 1$

$\sigma = \frac{2\pi^5 k_B^4}{15 c^2 h^3}$
$$$$
$\varepsilon(\nu) + \rho(\nu) + \tau(\nu) = 1$

$\varepsilon(\nu) = \alpha(\nu) $  \hfill Kirchoffs lag


$\varphi = \varepsilon \sigma T^4 \unit{\watt\per\meter^2}$  \hfill Strålningstäthet


$\Phi = A \varepsilon \sigma T^4 \unit{\watt}$  \hfill Strålningsintensitet


$\frac{h\nu_{max}}{k_B T} = 2.821$  \hfill Wiens förskjutningslag frekvens

% $\frac{hc}{\lambda_{max}k_B T} = 4.965$  \hfill Wiens förskjutningslag våglängd \textcolor{orange}{skippa?}

$\lambda_{max}T = 2.898\cdot 10^{-3} \SI{}{\meter\kelvin}$  \hfill Wiens förskjutningslag våglängd

$\displaystyle u(\nu, T) = \frac{8\pi h \nu^3}{c^3} \cdot \frac{1}{e^{\frac{h\nu}{k_BT}} - 1} \unit{\joule\second\per\cubic\meter}$  \hfill Planck-fördelningen


$\displaystyle U(T) = V \frac{\pi^5}{15} \cdot \frac{8h}{c^3}\left(\frac{k_BT}{h}\right)^4 \unit{\joule}$  \hfill Total energi hålrumsstrålning


$\varphi = \frac{1}{4V} U(T)c = \sigma T^4$  \hfill Strålningstäthet hålrumsstrålning


$E = h\nu = \frac{hc}{\lambda} \unit{\joule}$  \hfill Fotonenergi


\end{multicols}

\end{document}